\usepackage{graphicx}                              % 支持插入图片
\usepackage{ctex}
\AtBeginDocument{
\thispagestyle{empty}
\newgeometry{margin=0pt}
\noindent\includegraphics[width=\paperwidth,height=\paperheight,keepaspectratio]{三国演义.jpeg} % 封面图片
\restoregeometry
\clearpage
}                                                 % 封面设置
\setlength{\parindent}{2em}                       % 首行缩进两个字符
\usepackage{wrapfig}                              % 支持图文环绕
\linespread{1.15}                                 % 行距设置为1.15倍
\setlength{\textfloatsep}{2pt plus 2pt minus 2pt} % 文字与浮动体(如图)的上下间距
\setlength{\intextsep}{2pt plus 2pt minus 2pt}    % 环绕时图与文字的间距(更紧凑)
\setlength{\floatsep}{2pt plus 2pt minus 2pt}     % 浮动体之间的间距
\setlength{\abovecaptionskip}{2pt}                % 图题与图之间的间距
\setlength{\belowcaptionskip}{2pt}                % 图题与后续文字的间距
\renewcommand{\floatpagefraction}{0.9}            % 避免图单独占据过多页面
\renewcommand{\textfraction}{0.1}                 % 让页面尽量多放一些文字
\renewcommand{\thechapter}{\chinese{chapter}}       % 一级标题(#):一、二、三
\renewcommand{\thesection}{\chinese{section}}       % 二级标题(##):一、二、三
\renewcommand{\thesubsection}{\chinese{subsection}} % 三级标题(###):(一)、(二)
\renewcommand{\thesubsubsection}{\arabic{subsubsection}、} % 四级标题(####):1、2、3
\renewcommand{\theparagraph}{(\alph{paragraph})}     % 五级标题(#####):(a)、(b)、(c)
\renewcommand{\thesubparagraph}{\roman{subparagraph})} % 六级标题(######):i)、ii)、iii)
\usepackage[colorlinks=true,linkcolor=blue,citecolor=blue,urlcolor=blue]{hyperref}  % 为超链接设置颜色
